%---------------
% Question 3
%---------------
\begin{problem}{2-2}
\end{problem}

This type of problem is classified as a \textit{terminal control problem} in which a parameter
is being \textit{maximized} in which the final total volume of dye in tank 2 is to be as close
to $N$ ft\textsuperscript{3} as possible.

$$\\$$
\noindent \textbf{a)} \newline

A performance measure that can be used is

\begin{equation}
    J = - v_2 (t_f) \,.
\end{equation}

\noindent A minus sign is being used because the quantity $v_2(t_f)$ is being \textit{maximized}.

$$\\$$
\noindent \textbf{b)} \newline

A set of physically realizable state and control constraints are

\begin{align}
    0 \, \leq \, h_1 (t) \, & \leq H_1, \\
    0 \, \leq \, h_2 (t) \, & \leq H_2, \\
    0 \, \leq \, w_1 (t) \, & \leq W_1, \\
    0 \, \leq \, w_2 (t) \, & \leq W_2, \\
    \int_{t_0}^{t_f} m (t) \, dt \, & \leq N
\end{align}

\noindent where

\begin{itemize}
    \item $H_1$ and $H_2$ are the maximum heights of tanks 1 and 2, respectively
    \item $W_1$ and $W_2$ are the maximum rates of water entering tanks 1 and 2, respectively
    \item where $t_0$ and $t_f$ are the initial and final times, respectively, and $t_f \, - \, t_0 = 1$ day
\end{itemize}
